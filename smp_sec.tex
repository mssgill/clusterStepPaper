There are eight weak lensing pipelines that submitted 
results for the CSTEP project, shown in Table \ref{table:smp}.
There are several lensing pipelines tested in this challenge that could potentially
be used to create shear catalogs for DES. These implementations
represent a broad range of lensing pipeline types, and can be divided
into four groups. Methods are grouped using roughly the same criteria 
as in STEP2 \citep{STEP2}. A short description of the methods is
included below with a detailed pipeline description included in
Appendix \ref{App:shpipe}.\\

\subsection{Red class methods}
The red class methods (DE, IM, PK, KM) are based on the oldest 
shape measurement method KSB+ developed by Kaiser, Squires, and Broadhurst in 1995. As in STEP2, red class methods are
defined as those that measure combinations of
moments of each galaxy image $I(x)$ using a Gaussian 
weighting function. Various implementations
of this class of methods have been studied in previous shape
measurement challenges, and the accuracy has been shown 
to vary widely \citep{STEP2, GREAT10}.  Two of the red class lensing
pipeline implementations included in CSTEP have performed well on
previous image simulations challenges (DE, KM). Other 
red class implementations that were calibrated on
simulated image data have been used in several recent 
weak lensing analyses\citep[e.g.][]{Gruen_s, Apple, TS}. Many of the
large weak lensing cluster studies have used red class
methods \citep{HH}, and KSB+ methods are the most common
method used to measure the mass of clusters using weak lensing.
 
\subsection{Green class methods}
The green class methods (GM, I3) are commonly known as model fitting
methods. These methods convolve various models of galaxies with different 
parameterizations, including their intrinsic shape, with the PSF and
determine those that best fit the galaxy image. The green class methods
\textsc{lensfit} and \textsc{DeepZot} were among the best performing 
methods in GREAT08 and GREAT10, and im3shape has been shown 
to perform well on the GREAT08 and GREAT10 data \citep{Jzun}. 
A green class method \textsc{lensfit}, was 
the lensing pipeline for the Canada-France-Hawaii 
Telescope Lensing Survey (CFHTLenS) \citep{CHey}. This catalog 
was then used for several scientific analyses, \citep[e.g.][]{CH1,
  CH2, CH3}. 

\subsection{Blue class methods}
The blue class methods (MJ) are methods which model the
galaxy images as a sum of orthonormal Gauss-Laguerre polynomial
functions, commonly known as \textsc{shapelets}. The MJ
method competed in the STEP1, STEP2, and GREAT08 challenge. 


\subsection{Purple class methods}
The purple class is used to describe a method significantly different from the
three main approaches outlined above, Fourier Domain Null Testing (FDNT,
\citealt{Bern}). In FDNT, the surface brightness profiles of PSF and
observed galaxy are transformed to Fourier space, where the latter is
de-convolved by dividing by the former. Testing the Fourier
representation of the galaxy for roundness after applying an inverse
shear, a likelihood of shears is determined. The information used in
the roundness test is limited to frequencies below those rendered
infinitely noisy by the PSF. FDNT is known to perform very well on low
noise simulations \citep{Bern} but has not been applied yet to
observed data for a scientific analysis.
 
\begin{table}
\begin{center}
  \begin{tabular}{| c | c | c| c | }
    \hline 
     Method & Key & Contributor & Class \\
    \hline
     DEIMOS & DE &  P. Melchior & Red  \\
    \hline
    IMCAT & IM & J. Young & Red  \\
    \hline
    ksbm &  KM & P. Melchior &  Red \\
    \hline
    PKSB & PK & D. Gruen & Red \\
    \hline
     Gaussian Mixtures & GM & E. Sheldon & Green \\
    \hline
     im3shape &  I3 &  B. Rowe & Green \\
    \hline
     Bernstein and Jarvis (2002) & MJ &  M. Jarvis & Blue \\
    \hline
     PFDNT &  PF & D. Gruen & Purple \\
    \hline
  \end{tabular}
\end{center}
\caption{ A summary of the lensing pipelines used to analyze CSTEP simulated images. }
\label{table:smp}
\end{table}
