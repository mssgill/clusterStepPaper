Here we model the systematic errors on the stacked cluster weak
lensing measurement by comparing the average mass in each bin
to the mass measured by a shear profile, effected by shape 
measurement bias. To measure the effect of shape measurement bias, a Navarro, Frenk and White
(NFW) density profile is created for each mass bin. This density profile is
used to calculate the reduced shear $ g $ from the theoretical
predication as described in \cite{NFW}. \\
\indent The NWF density profile is given by
\begin{equation}
\rho(r) = \frac{\delta_c\rho_c}{(r/r_s)(1+ r/r_s)^{2}}
\end{equation}
where $\rho_c = \frac{3 H^2 (z) }{8 \pi G} $ is the critical density , $
H(z) $ is Hubble's parameter , $G$ is Newton's constant, $r_s =
r_{200}/c$, $c$ is the concentration and 
\begin{equation}
\delta_c = \frac{200}{3}\frac{c^3}{ln(1+c) - c/(1+c)}
\end{equation}
from \citep{NFW} . The reduced shear from a NFW
halo is 
\begin{equation}
g = \frac{\gamma}{1-\kappa} = \frac{ \Delta \Sigma / \Sigma_c }{1
-\overline{\Sigma}/ \Sigma_c}
\end{equation}
The reduced shear as measured by the current measurement pipelines is modeled as 
\begin{equation}
g' = g^2*Q + g*M
\end{equation}
We then fit this $ g' $ distribution to get a $M_{200}$ and $ c $
value. \\
\begin{figure}
\centering
\includegraphics[width=0.5\textwidth]{fig/stat_im3_halo.pdf} 
\caption{This figure compares the statistical and systematic error of
  for the stacked weak lensing measurement of the im3shape pipeline
  on DES.}
\label{fig:QMC_main_sel}
\end{figure}
